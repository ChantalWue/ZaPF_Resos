\documentclass[DIV=calc]{scrartcl}
\usepackage[utf8]{inputenc}
\usepackage[T1]{fontenc}
\usepackage[ngerman]{babel}
\usepackage{graphicx}
\usepackage[draft, markup=underlined]{changes}
\usepackage{csquotes}

\usepackage{ulem}
%\usepackage[dvipsnames]{xcolor}
\usepackage{paralist}
\usepackage{fixltx2e}
%\usepackage{ellipsis}
\usepackage[tracking=true]{microtype}

\usepackage{lmodern}                        % Ersatz fuer Computer Modern-Schriften
%\usepackage{hfoldsty}

%\usepackage{fourier}             % Schriftart
\usepackage[scaled=0.81]{helvet}     % Schriftart

\usepackage{url}
\usepackage{tocloft}             % Paket für Table of Contents

\usepackage{xcolor}
\definecolor{urlred}{HTML}{660000}

\usepackage{hyperref}
\hypersetup{
    colorlinks=true,    
    linkcolor=black,    % Farbe der internen Links (u.a. Table of Contents)
    urlcolor=black,    % Farbe der url-links
    citecolor=black} % Farbe der Literaturverzeichnis-Links

\usepackage{mdwlist}     % Änderung der Zeilenabstände bei itemize und enumerate
\usepackage{draftwatermark} % Wasserzeichen ``Entwurf'' 
\SetWatermarkText{}

\parindent 0pt                 % Absatzeinrücken verhindern
\parskip 12pt                 % Absätze durch Lücke trennen

\setlength{\textheight}{23cm}
\usepackage{fancyhdr}
\pagestyle{fancy}
\fancyhead{} % clear all header fields
\cfoot{}
\lfoot{Zusammenkunft aller Physik-Fachschaften}
\rfoot{www.zapfev.de\\stapf@zapf.in}
\renewcommand{\headrulewidth}{0pt}
\renewcommand{\footrulewidth}{0.1pt}
\newcommand{\gen}{*innen}
\addto{\captionsngerman}{\renewcommand{\refname}{Quellen}}

%%%% Mit-TeXen Kommandoset
\usepackage[normalem]{ulem}
\usepackage{xcolor}

\newcommand{\replace}[2]{
    \sout{\textcolor{blue}{#1}}~\textcolor{blue}{#2}
}

\newcommand{\delete}[1]{
    \sout{\textcolor{red}{#1}}
}

\newcommand{\add}[1]{
    \textcolor{blue}{#1}
}



\begin{document}
    \hspace{0.87\textwidth}
    \begin{minipage}{120pt}
        \vspace{-1.8cm}
        \includegraphics[width=80pt]{../../logo.pdf}
        \centering
        \small Zusammenkunft aller Physik-Fachschaften
    \end{minipage}
    \begin{center}
        \huge{Positionspapier der Zusammenkunft aller Physik-Fachschaften}\vspace{.25\baselineskip}\\
        \normalsize
    \end{center}
    \vspace{1cm}

\section*{Zur Förderung der Wissenschaftskommunikation in der akademischen Ausbildung}


Dieses Positionspapier ersetzt das Positionspapier des gleichen Titels, das auf der Winter-ZaPF 2017 in Siegen beschlossen wurde.\\~\\
Die Zusammenkunft aller Physikfachschaften (ZaPF) ist der Meinung, dass Wissenschaftskommunikation ein elementarer Bestandteil im Studium sein sollte. Wir sehen dafür unter anderem folgende Stellen im Bachelor- sowie Masterstudium, bei denen Wissenschaftskommunikation integriert werden kann:\\
\paragraph{Vortrag der Abschlussarbeiten:} Die ZaPF empfiehlt als Maßnahme, Angebote zu schaffen, um das Thema der eigenen Abschlussarbeit neben einer möglichen Verteidigung vorstellen zu können, um die Kompetenz, Wissenschaft zu kommunizieren, zu stärken. Sie ist der Meinung, dass hierfür beispielsweise Institutskolloquien, einen populärwissenschaftlichen Blogbeiträg veröffentlichen, Konferenzvorträge, Science Slams o. Ä. sinnvoll sind. Bei diesen kann aufgrund des diverseren Publikums die zielgruppenorientierte Kommunikation besser geübt werden als bei einem Publikum mit gleicher Spezialisierung. Insbesondere für die Masterarbeit wird deshalb eine Öffnung für die Allgemeinheit sehr empfohlen. 

\paragraph{Eigenständiges Modul oder Integration als Schlüsselkompetenz:}
Die ZaPF empfiehlt das Angebot einer Veranstaltung, die folgende theoretische und praktische Kompetenzen im Bereich der Wissenschaftskommunikation vermittelt:
\begin{itemize}

    \item Kenntnis von Konzepten der Wissenschaftskommunikation sowie Anwendung jener,

    \item Kenntnis und Anwendung diverser, über Präsentationen hinausgehender wissenschaftskommunikativer Formate,

    \item (populär-)wissenschaftliches Schreiben,

    \item Strukturierung wissenschaftlicher Inhalte sowie zielgruppenorientierte Aufbereitung dieser,

    \item selbstständige Darstellung eigener Forschung und

    \item Medienkompetenz, insbesondere Nutzung, Anwendung, Gestaltung und Einsatz von Medien.

\end{itemize}

Diese Veranstaltung sollte mindestens im Wahlpflichtbereich vorkommen. Sinnvoll für die Umsetzung erachtet die ZaPF ein Seminar und/oder eine Ringvorlesung mit folgenden beispielsweisen Inhalten:

\begin{itemize}

    \item Überblick über Konzepte und Umsetzungsmöglichkeiten der Wissenschaftkommunikation,

    \item herkömmliche Kommunikationsformate, z.B. Vorträge, Artikel in Zeitschriften, Fernseh- und Radiobeiträge, Tage der offenen Tür o.Ä., sowie

    \item alternative Kommunikationsformate, z.B. Blogs, Videos, Podcasts, Ausstellungen, Science Slams, interaktive Veranstaltungen, Nächte der Wissenschaft o.Ä.,

    \item Rhetorikschulung,

    \item Medientheorie und

    \item Darstellung der eigenen Forschung.

\end{itemize}

Ein fakultätenübergreifendes Modul wird ermutigt. Dessen Leitung kann sowohl von universitären Lehrkräften unterschiedlicher Fachbereiche\footnote{bspw. Physik, Germanistik, Journalismus, ...} als auch Mitarbeiter*innen zentraler Einrichtungen\footnote{bspw. Pressestelle, Kommunikationsbeauftragten, ...} oder externen Expert*innen übernommen werden. Die aus der Umsetzung des vorgeschlagenen Konzeptes resultierende Vernetzung von Studierenden mit anderen Fachbereichen und in der Forschung ist nur eine der positiven Auswirkungen. \\

Bis zum Erreichen des Masterabschlusses sollte mindestens eine solche Maßnahme durchgeführt worden sein. Die Einbindung dieses Themengebietes in das Curriculum wird gefordert, um sowohl die Akzeptanz und Wertschätzung von Wissenschaftskommunikation allgemein, als auch die Identifikation von Studierenden mit Forschung sowie die Interdisziplinarität zu fördern.
\vfill
    \begin{flushright}
        Verabschiedet am 25.11.2018 in Würzburg
    \end{flushright}
\end{document}

