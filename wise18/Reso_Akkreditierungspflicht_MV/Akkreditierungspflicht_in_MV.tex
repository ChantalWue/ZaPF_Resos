\documentclass[DIV=calc]{scrartcl}
\usepackage[utf8]{inputenc}
\usepackage[T1]{fontenc}
\usepackage[ngerman]{babel}
\usepackage{graphicx}
\usepackage[draft, markup=underlined]{changes}
\usepackage{csquotes}

\usepackage{ulem}
%\usepackage[dvipsnames]{xcolor}
\usepackage{paralist}
\usepackage{fixltx2e}
%\usepackage{ellipsis}
\usepackage[tracking=true]{microtype}

\usepackage{lmodern}                        % Ersatz fuer Computer Modern-Schriften
%\usepackage{hfoldsty}

%\usepackage{fourier}             % Schriftart
\usepackage[scaled=0.81]{helvet}     % Schriftart

\usepackage{url}
\usepackage{tocloft}             % Paket für Table of Contents

\usepackage{xcolor}
\definecolor{urlred}{HTML}{660000}

\usepackage{hyperref}
\hypersetup{
    colorlinks=true,    
    linkcolor=black,    % Farbe der internen Links (u.a. Table of Contents)
    urlcolor=black,    % Farbe der url-links
    citecolor=black} % Farbe der Literaturverzeichnis-Links

\usepackage{mdwlist}     % Änderung der Zeilenabstände bei itemize und enumerate
%\usepackage{draftwatermark} % Wasserzeichen ``Entwurf'' 
%\SetWatermarkText{Entwurf}

\parindent 0pt                 % Absatzeinrücken verhindern
\parskip 12pt                 % Absätze durch Lücke trennen

\setlength{\textheight}{23cm}
\usepackage{fancyhdr}
\pagestyle{fancy}
\fancyhead{} % clear all header fields
\cfoot{}
\lfoot{Zusammenkunft aller Physik-Fachschaften}
\rfoot{www.zapfev.de\\stapf@zapf.in}
\renewcommand{\headrulewidth}{0pt}
\renewcommand{\footrulewidth}{0.1pt}
\newcommand{\gen}{*innen}
\addto{\captionsngerman}{\renewcommand{\refname}{Quellen}}

%%%% Mit-TeXen Kommandoset
\usepackage[normalem]{ulem}
\usepackage{soul}
\usepackage{xcolor}

\newcommand{\replace}[2]{
    \sout{#1} \textcolor{blue}{#2}}
\newcommand{\delete}[1]{
    %\sout{\textcolor{red}{#1}}
    \sout{#1}}
\newcommand{\add}[1]{
    \textcolor{blue}{#1}}


\begin{document}
    \hspace{0.87\textwidth}
    \begin{minipage}{120pt}
        \vspace{-1.8cm}
        \includegraphics[width=80pt]{../../logo.pdf}
        \centering
        \small Zusammenkunft aller Physik-Fachschaften
    \end{minipage}
    \begin{center}
        \huge{Resolution der Zusammenkunft aller Physik-Fachschaften}\vspace{.25\baselineskip}\\
        \normalsize
    \end{center}
    \vspace{1cm}

\section*{Zur Akkreditierungspflicht von Studiengängen in Mecklenburg-Vorpommern}

Die ZaPF betrachtet mit Sorge die Bestrebungen der Landesregierung \linebreak[4] Mecklenburg-Vorpommern, die Akkreditierungspflicht für Studiengänge im Zuge der Novellierung des Hochschulgesetzes abzuschaffen. Die Akkreditierung hat sich als Mittel der Qualitätssicherung bewährt. Sie ist ein bundesweiter Standard und europaweit anerkannt. Weiterhin sichert sie die Teilhabe verschiedener Parteien, insbesondere der Studierenden, an Qualitätssicherungsverfahren und hilft, einen einheitlichen Mindeststandard für den Aufbau von Studiengängen deutschlandweit zu etablieren.

Aufgrund ihrer weiten Verbreitung verlassen sich viele Arbeitgebende auf die Existenz akkreditierter Studiengänge. Durch den Wegfall dieses Merkmals in Mecklenburg-Vorpommern können Nachteile bei der Arbeitsplatzsuche für Absolvent*innen dieser Studiengänge entstehen und mecklenburg-vorpommerische Hochschulen werden zunehmend unattraktiver für Studieninteressierte und Studierende. Damit wird mutwillig in Kauf genommen, dass Studierendenzahlen sinken, Studierendenmobilität eingeschränkt wird und die Hochschulen an Reputation verlieren.

Aus diesen Gründen fordern wir die Beibehaltung der Akkreditierungspflicht.
\vfill
    \begin{flushright}
        Verabschiedet am 25.11.2018 in Würzburg
    \end{flushright}
\end{document}

